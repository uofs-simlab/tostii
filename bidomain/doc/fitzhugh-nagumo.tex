\documentclass{article}

\usepackage{amsfonts}
\usepackage{amsmath}
\usepackage{mathrsfs}

\newcommand{\bvec}[1]{\boldsymbol{#1}}
\newcommand{\brvec}[1]{\mathbf{#1}}
\newcommand{\bmat}[1]{\boldsymbol{#1}}
\newcommand{\brmat}[1]{\mathbf{#1}}
\newcommand{\ii}{\mathrm{i}}
\newcommand{\ee}{\mathrm{e}}
\newcommand{\dd}{\mathrm{d}}

\begin{document}

\section{Introduction}

We consider the Bidomain equations:
\begin{align*}
    \chi C_\text{m} \frac{\partial v}{\partial t}
            + \chi (I_\text{ion}(v) + I_\text{stim})
        & = \nabla \cdot (\sigma_\text{i} \nabla (v + u_\text{e})) \\
    0
        & = \nabla \cdot (\sigma_\text{i} \nabla (v + u_\text{e}))
            + \nabla \cdot (\sigma_\text{e} \nabla u_\text{e})
\end{align*}
\noindent We will be considering Neumann boundary conditions only.
The domain is $\Omega = [0, 1]^n$.

We introduce a new state variable $w$,
and the FitzHugh-Nagumo cell model:
\begin{align*}
    I_\text{ion}(v, w)
        & = \varepsilon^{-1} (v - \frac{v^3}{3} - w) \\
    \frac{\partial w}{\partial t}
        & = \varepsilon (v + \beta - \gamma w)
\end{align*}
\noindent When we add this cell model into the original equations,
we have the new system:
\begin{align*}
    \chi C_\text{m} \frac{\partial v}{\partial t}
            + \chi \varepsilon^{-1} (v - \frac{v^3}{3} - w)
            + \chi I_\text{stim}
        & = \nabla \cdot (\sigma_\text{i} \nabla (v + u_\text{e})) \\
    \frac{\partial w}{\partial t}
        & = \varepsilon (v + \beta - \gamma w) \\
    0
        & = \nabla \cdot (\sigma_\text{i} \nabla (v + u_\text{e}))
            + \nabla \cdot (\sigma_\text{e} \nabla u_\text{e})
\end{align*}
\noindent We will be splitting the above system of equations into an explicit operator
involving $I_\text{ion}$ and $I_\text{stim}$:
\begin{align*}
    \chi C_\text{m} \frac{\partial v}{\partial t}
            + \chi \varepsilon^{-1} (v - \frac{v^3}{3} - w)
            + \chi I_\text{stim}
        & = 0 \\
    \frac{\partial w}{\partial t}
        & = \varepsilon (v + \beta - \gamma w) \\
    0 & = 0
\end{align*}
\noindent And an implicit operator involving the Laplacian terms:
\begin{align*}
    \chi C_\text{m} \frac{\partial v}{\partial t}
        & = \nabla \cdot (\sigma_\text{i} \nabla (v + u_\text{e})) \\
    \frac{\partial w}{\partial t} & = 0 \\
    0
        & = \nabla \cdot (\sigma_\text{i} \nabla (v + u_\text{e}))
            + \nabla \cdot (\sigma_\text{e} \nabla u_\text{e})
\end{align*}

\newpage
\section{Spatial Discretization}

Consider the monolithic system of equations,
and let $\bvec{q} = \begin{bmatrix}
    v \\
    w \\
    u_\text{e}
\end{bmatrix}$.
We consider a set of basis functions $\bvec{\Phi}_i : \Omega \rightarrow \mathbb{R}^3$,
where each $\bvec{\Phi}_i = \varphi_{b_i} \hat{\brvec{e}}_{c_i}$
for some basis functions $\varphi_i : \Omega \rightarrow \mathbb{R}$.
We make the approximation $\bvec{q} = Q_j \bvec{\Phi}_j$ (Einstein summation)
for a vector $\bvec{Q} \in \mathbb{R}^N$, where $N$ is the number of degrees of freedom.
Also, we approximate $\frac{\partial \bvec{q}}{\partial t} = \dot{Q}_j \bvec{\Phi}_j$.
We multiply the monolithic system of equations by a test function $\bvec{\Phi}_i$ and integrate:
\begin{align*}
    \left< \bvec{\Phi}_i \mid \begin{matrix}
            \chi C_\text{m} \partial_t v \\
            \partial_t w \\
            0
        \end{matrix} \right>
        & = \left< \bvec{\Phi}_i \mid \begin{matrix}
                -\chi \varepsilon^{-1} (v - \frac{v^3}{3} - w)
                    - \chi I_\text{stim} \\
                \varepsilon (v + \beta - \gamma w) \\
                0
            \end{matrix} \right> \\
        \qquad & + \left< \bvec{\Phi}_i \mid \begin{matrix}
                \nabla \cdot (\sigma_\text{i} \nabla (v + u_\text{e})) \\
                0 \\
                \nabla \cdot (\sigma_\text{i} \nabla (v + u_\text{e}))
                    + \nabla \cdot (\sigma_\text{e} \nabla u_\text{e})
            \end{matrix} \right>
\end{align*}
\noindent Using the fact that $v = Q_j \varphi_{b_j} \delta_{1, c_j}$,
$w = Q_j \varphi_{b_j} \delta_{2, c_j}$,
and $u_\text{e} = Q_j \varphi_{b_j} \delta_{3, c_j}$,
as well as $\frac{\partial v}{\partial t} = \dot{Q}_j \varphi{b}_j \delta_{1, c_j}$
and $\frac{\partial w}{\partial t} = \dot{Q}_j \varphi_{b_j} \delta_{2, c_j}$,
we consider each of the forms above:
\begin{align*}
    & \left< \bvec{\Phi}_i \mid \begin{matrix}
            \chi C_\text{m} \partial_t v \\
            \partial_t w \\
            0
        \end{matrix} \right> \\
    & \qquad = \delta_{1, c_i} \left< \varphi_{b_i} \mid \chi C_\text{m} \partial_t v \right>
        + \delta_{2, c_i} \left< \varphi_{b_i} \mid \partial_t w \right> \\
    & \qquad = \chi C_\text{m} \delta_{1, c_i} \delta_{1, c_j} \left< \varphi_{b_i} \mid \dot{Q}_j \varphi_{b_j} \right>
        + \delta_{2, c_i} \delta_{2, c_j} \left< \varphi_{b_i} \mid \dot{Q}_j \varphi_{b_j} \right> \\
    & \qquad = (\chi C_\text{m} \delta_{1, c_i} \delta_{1, c_j} + \delta_{2, c_i} \delta_{2, c_j})
        \left< \varphi_{b_i} \mid \varphi_{b_j} \right> \dot{Q}_j
\end{align*}
\begin{align*}
    & \left< \bvec{\Phi}_i \mid \begin{matrix}
            -\chi \varepsilon^{-1} (v - \frac{v^3}{3} - w)
                - \chi I_\text{stim} \\
            \varepsilon (v + \beta - \gamma w) \\
            0
        \end{matrix} \right> \\
    & \qquad = \delta_{1, c_i} \left< \varphi_{b_i} \mid -\chi \varepsilon^{-1} (v - \frac{v^3}{3} - w)
        - \chi I_\text{stim} \right>
        + \delta_{2, c_i} \left< \varphi_{b_i} \mid \varepsilon (v + \beta - \gamma w) \right> \\
    & \qquad = (-\chi \varepsilon^{-1} \delta_{1, c_i} + \varepsilon \delta_{2, c_i})
            \left< \varphi_{b_i} \mid v \right>
        + (\chi \varepsilon^{-1} \delta_{1, c_i} - \varepsilon \gamma \delta_{2, c_i})
            \left< \varphi_{b_i} \mid w \right> \\
        & \qquad \qquad + \frac{1}{3} \chi \varepsilon^{-1} \delta_{1, c_i}
            \left< \varphi_{b_i} \mid v^3 \right>
        - \chi \delta_{1, c_i} \left< \varphi_{b_i} \mid I_\text{stim} \right>
        + \varepsilon \beta \delta_{2, c_i} \left< \varphi_{b_i} \mid 1 \right> \\
    & \qquad = (-\chi \varepsilon^{-1} \delta_{1, c_i} + \varepsilon \delta_{2, c_i}) \delta_{1, c_j}
            \left< \varphi_{b_i} \mid Q_j \varphi_{b_j} \right>
        + (\chi \varepsilon^{-1} \delta_{1, c_i} - \varepsilon \gamma \delta_{2, c_i}) \delta_{2, c_j}
            \left< \varphi_{b_i} \mid Q_j \varphi_{b_j} \right> \\
        & \qquad \qquad + \frac{1}{3} \chi \varepsilon^{-1} \delta_{1, c_i}
            \left< \varphi_{b_i} \mid v^3 \right>
        - \chi \delta_{1, c_i} \left< \varphi_{b_i} \mid I_\text{stim} \right>
        + \varepsilon \beta \delta_{2, c_i} \left< \varphi_{b_i} \mid 1 \right> \\
    & \qquad = (-\chi \varepsilon^{-1} \delta_{1, c_i} \delta_{1, c_j} + \varepsilon \delta_{2, c_i} \delta_{1, c_j}
                + \chi \varepsilon^{-1} \delta_{1, c_i} \delta_{2, c_j} - \varepsilon \gamma \delta_{2, c_i} \delta_{2, c_j})
            \left< \varphi_{b_i} \mid \varphi_{b_j} \right> Q_j \\
        & \qquad \qquad + \frac{1}{3} \chi \varepsilon^{-1} \delta_{1, c_i}
            \left< \varphi_{b_i} \mid v^3 \right>
        - \chi \delta_{1, c_i} \left< \varphi_{b_i} \mid I_\text{stim} \right>
        + \varepsilon \beta \delta_{2, c_i} \left< \varphi_{b_i} \mid 1 \right>
\end{align*}
\begin{align*}
    & \left< \bvec{\Phi}_i \mid \begin{matrix}
            \nabla \cdot (\sigma_\text{i} \nabla (v + u_\text{e})) \\
            0 \\
            \nabla \cdot (\sigma_\text{i} \nabla (v + u_\text{e}))
                + \nabla \cdot (\sigma_\text{e} \nabla u_\text{e})
        \end{matrix} \right> \\
    & \qquad = \delta_{\{1,3\}, c_i} \left< \varphi_{b_i} \mid \nabla \cdot (\sigma_\text{i} \nabla (v + u_\text{e})) \right>
        + \delta_{3, c_i} \left< \varphi_{b_i} \mid \nabla \cdot (\sigma_\text{e} \nabla u_\text{e}) \right> \\
    & \qquad = -\delta_{\{1,3\}, c_i} \left< \nabla \varphi_{b_i} \mid \sigma_i \nabla (v + u_\text{e}) \right>
        - \delta_{3, c_i} \left< \nabla \varphi_{b_i} \mid \sigma_\text{e} \nabla u_\text{e} \right> \\
    & \qquad = -\sigma_\text{i} \delta_{\{1,3\}, c_i}
            \left< \nabla \varphi_{b_i} \mid \nabla (Q_j \varphi_{b_j} \delta_{\{1,3\}, c_j}) \right>
        - \sigma_\text{e} \delta_{3, c_i} \left< \nabla \varphi_{b_i} \mid \nabla (Q_j \varphi_{b_j} \delta_{3, c_j}) \right> \\
    & \qquad = -(\sigma_\text{i} \delta_{\{1,3\}, c_i} \delta_{\{1,3\}, c_j} + \sigma_\text{e} \delta_{3, c_i} \delta_{3, c_j})
            \left< \nabla \varphi_{b_i} \mid \nabla \varphi_{b_j} \right> Q_j
\end{align*}
\noindent So, define:
\begin{align*}
    M_{ij} & = (\chi C_\text{m} \delta_{1, c_i} \delta_{1,c_j} + \delta_{2, c_i} \delta_{2, c_j})
        \left< \varphi_{b_i} \mid \varphi_{b_j} \right> \\
    A_{ij} & = (-\chi \varepsilon^{-1} \delta_{1, c_i} \delta_{1, c_j} + \varepsilon \delta_{2, c_i} \delta_{1, c_j}
            + \chi \varepsilon^{-1} \delta_{1, c_i} \delta_{2, c_j} - \varepsilon \gamma \delta_{2, c_i} \delta_{2, c_j})
        \left< \varphi_{b_i} \mid \varphi_{b_j} \right> \\
    f_i & = -\chi \delta_{1, c_i} \left< \varphi_{b_i} \mid I_\text{stim} \right>
        + \varepsilon \beta \delta_{2, c_i} \left< \varphi_{b_i} \mid 1 \right> \\
    B_{ij} & = -(\sigma_\text{i} \delta_{\{1,3\}, c_i} \delta_{\{1,3\}, c_j} + \sigma_\text{e} \delta_{3, c_i} \delta_{3, c_j})
        \left< \nabla \varphi_{b_i} \mid \nabla \varphi_{b_j} \right>
\end{align*}
\noindent Then, we rewrite the monolithic equation as:
\begin{align*}
    \bmat{M} \dot{\bvec{Q}}
        = \bmat{A} \bvec{Q}
            + \frac{1}{3} \chi \varepsilon^{-1} \delta_{1, c_i} \left< \varphi_{b_i} \mid v^3 \right>
            + \bvec{f}
            + \bmat{B} \bvec{Q}
\end{align*}
\noindent And, the split equations become:
\begin{align*}
    \bmat{M} \dot{\bvec{Q}}
        & = \bmat{A} \bvec{Q}
            + \frac{1}{3} \chi \varepsilon^{-1} \delta_{1, c_i} \left< \varphi_{b_i} \mid v^3 \right>
            + \bvec{f} \\
    \bmat{M} \dot{\bvec{Q}}
        & = \bmat{B} \bvec{Q}
\end{align*}

\newpage
\section{Time Integration}

As a DAE, complications arise when integrating the Bidomain equations
because the mass matrix is singular.
Our goal is to solve the first ("explicit") equation using an explicit method such as Forward Euler;
we solve the second ("implicit") equation using an implicit method such as Backward Euler.

Recall that matrix form of the explicit equation from the previous section:
\begin{align*}
    \bmat{M} \dot{\bvec{Q}}
        = \bmat{A} \bvec{Q}
            + \frac{1}{3} \chi \varepsilon^{-1} \delta_{1, c_i}
                \left< \varphi_{b_i} \mid v^3 \right>
            + \bvec{f}
\end{align*}
\noindent To integrate this system, we write the equation as:
\begin{align*}
    \dot{\bvec{Q}}
        & = \bvec{F}_\text{E}(t, \bvec{Q}) \\
        & = \bmat{M}^{-1}
            (\bmat{A} \bvec{Q}
                + \frac{1}{3} \chi \varepsilon^{-1} \delta_{1, c_i}
                    \left< \varphi_{b_i} \mid v^3 \right>
                + \bvec{f})
\end{align*}
\noindent The solution $\bvec{Q}_{t + 1}$ at time step $t + 1$
is then given by $\bvec{Q}_t$ plus a weighted sum of stages
$\bvec{F}_\text{E}(t^{(i)}, \bvec{Q}_t^{(i)})$,
where each $\bvec{Q}_t^{(i)}$ is itself a weighted sum of previously computed stages.
However, $\bmat{M}$ is singular
(because no time derivatives of $u_\text{e}$ appear in the Bidomain equations),
and all components of the explicit equation's right-hand side are zero
(because the operator has been split such that the explicit equation does not involve $u_\text{e}$).

Denote the set of all DoF indices by $\mathcal{I} = [0, N)$.
Then, let $\mathcal{J} = \{\, i \in \mathcal{I} \mid c_i \in \{\,1, 2\,\} \,\}$
be the set of indices corresponding to the components $v$ and $w$. Then:
\begin{align*}
    \bmat{M}_\mathcal{J} \dot{\bvec{Q}}_\mathcal{J}
        = \bmat{A}_\mathcal{J} \bvec{Q}_\mathcal{J}
            + \frac{1}{3} \chi \varepsilon^{-1} \delta_{1, c_i}
                \left< \varphi_{b_i} \mid v^3 \right>
            + \bvec{f}_\mathcal{J}
\end{align*}
\noindent Where a matrix or vector subscripted with $\mathcal{J}$
denotes that only those elements with indices in $\mathcal{J}$ are included.
Because there are no longer elements corresponding to the $u_\text{e}$ component
of the system, $\bmat{M}_\mathcal{J}$ is non-singular, and we may write:
\begin{align*}
    \bvec{F}_\text{E}(t, \bvec{Q}_\mathcal{J})
        = \bmat{M}_\mathcal{J}^{-1}
            (\bmat{A}_\mathcal{J} \bvec{Q}_\mathcal{J}
                + \frac{1}{3} \chi \varepsilon^{-1} \delta_{1, c_i}
                    \left< \varphi_{b_i} \mid v^3 \right>
                + \bvec{f}_\mathcal{J})
\end{align*}

Now, we recall the matrix form of the implicit equation:
\begin{align*}
    \bmat{M} \dot{\bvec{Q}} = \bmat{B} \bvec{Q}
\end{align*}
Rather than formulating the solution of each time step as a sum of terms,
we shall derive an explicit equation that can be solved,
corresponding to the Backward Euler and Crank Nicolson methods.
First, we again rearrange the equation:
\begin{align*}
    \dot{\bvec{Q}}
        & = \bvec{F}_\text{I}(t, \bvec{Q}) \\
        & = \bmat{M}^{-1} \bmat{B} \bvec{Q}
\end{align*}
\noindent For Backward Euler:
\begin{align*}
    \bvec{Q}_{t + 1} & = \bvec{Q}_t + h \bvec{F}_\text{I}(t + h, \bvec{Q}_{t + 1}) \\
    \bvec{Q}_{t + 1} - \bvec{Q}_t & = h \bmat{M}^{-1} \bmat{B} \bvec{Q}_{t + 1} \\
    \bmat{M} (\bvec{Q}_{t + 1} - \bvec{Q}_t) & = h \bmat{B} \bvec{Q}_{t + 1} \\
    (\bmat{M} - h \bmat{B}) \bvec{Q}_{t + 1} & = \bmat{M} \bvec{Q}_t
\end{align*}
\noindent And, for Crank Nicolson:
\begin{align*}
    \bvec{Q}_{t + 1} & = \bvec{Q}_t + \frac{h}{2}
        (\bvec{F}_\text{I}(t, \bvec{Q}_t) + \bvec{F}_\text{I}(t + h, \bvec{Q}_{t + 1})) \\
    \bvec{Q}_{t + 1} - \bvec{Q}_t & = \frac{h}{2}
        (\bmat{M}^{-1} \bmat{B} \bvec{Q}_t + \bmat{M}^{-1} \bmat{B} \bvec{Q}_{t + 1}) \\
    \bmat{M} (\bvec{Q}_{t + 1} - \bvec{Q}_t) & = \frac{h}{2}
        (\bmat{B} \bvec{Q}_t + \bmat{B} \bvec{Q}_{t + 1}) \\
    (\bmat{M} - \frac{h}{2} \bmat{B}) \bvec{Q}_{t + 1} & = (\bmat{M} + \frac{h}{2} \bmat{B}) \bvec{Q}_t
\end{align*}
\noindent We generalize by picking some $\theta \in (0, 1]$:
\begin{align*}
    (\bmat{M} - \theta h \bmat{B}) \bvec{Q}_{t + 1} = (\bmat{M} + (1 - \theta) h \bmat{B}) \bvec{Q}_t
\end{align*}
\noindent Where Backward Euler corresponds to $\theta = 1$
and Crank Nicolson corresponds to $\theta = \frac{1}{2}$.

Furthermore, we can pull the same trick as with the explicit equation
and define $\mathcal{K} = \{\, i \in \mathcal{I} \mid c_i \in \{\, 1, 3 \,\} \,\}$.
Note that in the implicit equation, the only involvement of $w$ is
the equation $\frac{\partial w}{\partial t} = 0$,
so $w$ components (with $c_i = 2$) can be safely ignored.
This yields the update formula:
\begin{align*}
    (\bmat{M}_\mathcal{K} - \theta h \bmat{B}_\mathcal{K}) \bvec{Q}_{\mathcal{K}, t + 1}
        = (\bmat{M}_\mathcal{K} + (1 - \theta) h \bmat{B}_\mathcal{K}) \bvec{Q}_{\mathcal{K}, t}
\end{align*}

\end{document}