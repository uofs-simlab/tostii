\documentclass{article}

\usepackage{amsfonts}
\usepackage{amsmath}

\newcommand{\bvec}[1]{\boldsymbol{#1}}
\newcommand{\brvec}[1]{\mathbf{#1}}
\newcommand{\bmat}[1]{\boldsymbol{#1}}
\newcommand{\brmat}[1]{\mathbf{#1}}
\newcommand{\ii}{\mathrm{i}}
\newcommand{\ee}{\mathrm{e}}
\newcommand{\dd}{\mathrm{d}}

\begin{document}

\section{Introduction}

This document is relevant to the targets:
\begin{itemize}
    \item \texttt{bidomain}
    \item \texttt{bidomain\_godunov\_2}
    \item \texttt{bidomain\_godunov}
    \item \texttt{bidomain\_godunov\_2}
\end{itemize}
\noindent The section "Toy Problem" is relevant to
\texttt{bidomain} and \texttt{bidomain\_godunov},
while "Toy Problem 2" is relevant to
\texttt{bidomain\_2} and \texttt{bidomain\_godunov\_2}.
The section "Time Integration" is relevant to
\texttt{bidomain} and \texttt{bidomain\_2}.
Naturally, \texttt{bidomain\_godunov} and \texttt{bidomain\_godunov\_2}
rely on the section "Godunov Splitting";
this section makes the same assumption and method
as the previous section, "Linear Formulation".
See the section "Version History" for more information
about these different targets.

Let:
\begin{align*}
    n & \in \mathbb{N} \\
    T & \in \mathbb{R} \\
    \Omega & \subseteq \mathbb{R}^n \\
    \chi, C_m & \in \mathbb{R} \\
    \bmat{\sigma}_\text{i}, \bmat{\sigma}_\text{e} & \in \mathbb{R}^{n \times n} \\
    I_\text{ion} & : \mathbb{R} \rightarrow \mathbb{R} \\
    f, f_\text{e} & : (0, T) \times \Omega \rightarrow \mathbb{R} \\
    v, u_\text{e} & : (0, T) \times \Omega \rightarrow \mathbb{R}
\end{align*}
Denote $\bvec{w}(t, \bvec{x}) = \begin{bmatrix}
    v \\
    u_\text{e}
\end{bmatrix}$. The Bidomain problem is given as:
\begin{align} \label{eq1}
    \chi C_\text{m} \frac{\partial v}{\partial t}
        + \chi I_\text{ion}(v)
        & = \nabla \cdot (\bmat{\sigma}_\text{i} \nabla (v + u_\text{e}))
        + f \\
    \label{eq2}
    0
        & = \nabla \cdot (\bmat{\sigma}_\text{i} \nabla (v + u_\text{e}))
        + \nabla \cdot (\bmat{\sigma}_\text{e} \nabla u_\text{e})
        + f_\text{e}
\end{align}
\noindent With initial conditions:
\begin{equation}
    \bvec{w}(0, \bvec{x}) = \bvec{w}_0(\bvec{x})
\end{equation}
\noindent For all $\bvec{x} \in \Omega$ and with boundary conditions:
\begin{equation} \label{BCs}
    \bmat{\sigma}_\text{i} \nabla (v + u_\text{e}) \cdot \hat{\brvec{n}}
        = \bmat{\sigma}_\text{e} \nabla u_\text{e} \cdot \hat{\brvec{n}}
        = 0
\end{equation}
\noindent For all $(t, \bvec{x}) \in (0, T) \times \partial \Omega$.

Let $R_\text{m} \in \mathbb{R}$.
For a passive cell model, we take $I_\text{ion}(v) = \frac{1}{R_\text{m}} v$.
We also assume that $\bmat{\sigma}_\text{i} = \sigma_\text{i} \brmat{I}$
and $\bmat{\sigma}_\text{e} = \sigma_\text{e} \brmat{I}$
for some $\sigma_\text{i}, \sigma_\text{e} \in \mathbb{R}$,
so that $\bmat{\sigma}_\text{i}$ and $\bmat{\sigma}_\text{e}$
can be replaced with $\sigma_\text{i}$ and $\sigma_\text{e}$
in the equations above.

\newpage

\section{Toy Problem}

We consider a toy problem with $T = 1$, and $\Omega = [0, 1]^n$:
\begin{align}
    u_\text{e}(t, \bvec{x}) & = t^3 \prod_{i = 1}^n \cos(\pi x_i) \\
    v(t, \bvec{x}) & = -\frac{\sigma_\text{i} + \sigma_\text{e}}{\sigma_\text{i}} u_\text{e}(t, \bvec{x})
\end{align}
\noindent It is clear that $v(0, \bvec{x}) = u_\text{e}(0, \bvec{x}) = 0$, so we take $\bvec{w}_0(\bvec{x}) = \brvec{0}$.

Next, we consider:
\begin{align*}
    \sigma_\text{i} \nabla (v + u_\text{e})
        & = \sigma_\text{i} (\nabla v + \nabla u_\text{e}) \\
        & = \sigma_\text{i} (-\frac{\sigma_\text{i} + \sigma_\text{e}}{\sigma_\text{i}} \nabla u_\text{e} + \nabla u_\text{e}) \\
        & = -\sigma_\text{i} \nabla u_\text{e} - \sigma_\text{e} \nabla u_\text{e} + \sigma_\text{i} \nabla u_\text{e} \\
        & = -\sigma_\text{e} \nabla u_\text{e}
\end{align*}
\noindent Since $\frac{\partial u_\text{e}}{\partial x_k} = -\pi t^3 \sin(\pi x_k) \prod_{i \neq k} \cos(\pi x_i)$, we see that
$\frac{\partial u_\text{e}}{\partial x_k} = 0$ when $x_k \in \mathbb{Z}$.
Recall that $\Omega = [0, 1]^n$; then, on the $x_k$-faces of $\partial \Omega$,
we have that $\frac{\partial u_\text{e}}{\partial x_k} = 0$
and $\hat{\brvec{n}} = \pm \hat{\brvec{e}}_k$.
So, $\nabla u_\text{e} \cdot \hat{\brvec{n}} = 0$ for all $\bvec{x} \in \partial \Omega$,
and both boundary conditions in Equation \ref{BCs} are satisfied.

Then, we determine $f$ and $f_\text{e}$.
From above, we have that $\sigma_\text{i} \nabla (v + u_\text{e}) = -\sigma_\text{e} \nabla u_\text{e}$,
and so, rewriting Equation \ref{eq2}:
\begin{align*}
    f_\text{e} & = 
            - \nabla \cdot (\sigma_\text{i} \nabla (v + u_\text{e})) 
            - \nabla \cdot (\sigma_\text{e} \nabla u_\text{e}) \\
        & = \nabla \cdot (\sigma_\text{e} \nabla u_\text{e})
            - \nabla \cdot (\sigma_\text{e} \nabla u_\text{e}) \\
        & = 0
\end{align*}
\noindent When we note that:
\begin{align*}
    \frac{\partial^2 u_\text{e}}{\partial x_k^2} & =
            - \frac{\partial}{\partial x_k} \pi t^3 \sin(\pi x_k) \prod_{i \neq k} \cos(\pi x_i) \\
        & = - \pi^2 t^3 \prod_{i = 1}^n \cos(\pi x_i) \\
        & = - \pi^2 u_\text{e} \\
    \Delta u_\text{e} & =
            \sum_{k = 1}^n \frac{\partial^2 u_\text{e}}{\partial x_k^2} \\
        & = - \sum_{k = 1}^n \pi^2 u_\text{e} \\
        & = - n \pi^2 u_\text{e}
\end{align*}
\noindent We rewrite Equation \ref{eq1}:
\begin{align*}
    f & = \chi C_\text{m} \frac{\partial v}{\partial t}
            + \chi I_\text{ion}(v)
            - \nabla \cdot (\sigma_\text{i} \nabla (v + u_\text{e})) \\
        & = - \chi C_\text{m} \frac{\sigma_\text{i} + \sigma_\text{e}}{\sigma_\text{i}} \frac{\partial u_\text{e}}{\partial t}
            + \chi I_\text{ion}(v)
            - \nabla \cdot (-\sigma_\text{e} \nabla u_\text{e}) \\
        & = - \chi C_\text{m} \frac{\sigma_\text{i} + \sigma_\text{e}}{\sigma_\text{i}} \frac{3}{t} u_\text{e}
            - \chi \frac{1}{R_\text{m}} \frac{\sigma_\text{i} + \sigma_\text{e}}{\sigma_\text{i}} u_\text{e}
            + \sigma_\text{e} \Delta u_\text{e} \\
        & = - \chi (\frac{3 C_\text{m}}{t} + \frac{1}{R_\text{m}}) \frac{\sigma_\text{i} + \sigma_\text{e}}{\sigma_\text{i}} u_\text{e}
            - n \sigma_\text{e} \pi^2 u_\text{e} \\
        & = - (\chi \frac{\sigma_\text{i} + \sigma_\text{e}}{\sigma_\text{i}} (\frac{3 C_\text{m}}{t} + \frac{1}{R_\text{m}})
            + n \sigma_\text{e} \pi^2) u_\text{e}
\end{align*}

\newpage

\section{Toy Problem 2}

We will consider another toy problem,
to address the fact that $u_e = 0$
is a simple way to minimize the residual in the previous toy problem:
\begin{align*}
    \omega(\bvec{x}) & = \prod_{i = 1}^n \cos(\pi x_i) \\
    v(t, \bvec{x}) & = t^2 \omega(\bvec{x}) \\
    u_\text{e}(t, \bvec{x}) & = t \omega(\bvec{x})
\end{align*}
\noindent We see that:
\begin{align*}
    \Delta \omega & = -n \pi^2 \omega \\
    \Delta v & = -n \pi^2 v \\
    \Delta u_\text{e} & = -n \pi^2 u_\text{e}
\end{align*}
\noindent And:
\begin{align*}
    \frac{\partial v}{\partial t} & = \frac{2}{t} v \\
        & = 2 u_\text{e}
\end{align*}
\noindent So, rewriting the Bidomain equations:
\begin{align*}
    f
        & = \chi C_\text{m} \frac{\partial v}{\partial t}
            + \frac{\chi}{R_\text{m}} v
            - \nabla \cdot (\sigma_\text{i} \nabla (v + u_\text{e})) \\
        & = 2 \chi C_\text{m} u_\text{e}
            + \frac{\chi t}{R_\text{m}} u_\text{e}
            - \sigma_\text{i} \Delta v
            - \sigma_\text{i} \Delta u_\text{e} \\
        & = (2 \chi C_\text{m} + \frac{\chi t}{R_\text{m}}) u_\text{e}
            + n \pi^2 \sigma_\text{i} v
            + n \pi^2 \sigma_\text{i} u_\text{e} \\
        & = (2 \chi C_\text{m} + \frac{\chi t}{R_\text{m}} + n \pi^2 \sigma_\text{i}) u_\text{e}
            + n \pi^2 \sigma_\text{i} t u_\text{e} \\
        & = (2 \chi C_\text{m} + \frac{\chi t}{R_\text{m}} + n \pi^2 \sigma_\text{i} (t + 1)) u_\text{e} \\
    f_\text{e}
        & = -\nabla \cdot (\sigma_\text{i} \nabla (v + u_\text{e}))
            - \nabla \cdot (\sigma_\text{e} \nabla u_\text{e}) \\
        & = -\sigma_\text{i} \Delta v
            - (\sigma_\text{i} + \sigma_\text{e}) \Delta u_\text{e} \\
        & = n \pi^2 \sigma_\text{i} v
            + n \pi^2 (\sigma_\text{i} + \sigma_\text{e}) u_\text{e} \\
        & = n \pi^2 (\sigma_\text{i} t + \sigma_\text{i} + \sigma_\text{e}) u_\text{e}
\end{align*}

\newpage

\section{Spatial Discretization}

We consider a set of basis functions $\bvec{\Phi}_i : \mathbb{R}^n \rightarrow \mathbb{R}^2$,
where each $\bvec{\Phi}_i(\bvec{x}) = \varphi_{b_i} \hat{\brvec{e}}_{c_i}$
for some set of basis functions $\varphi_i : \mathbb{R} \rightarrow \mathbb{R}$
($b_i$ is the "base" of the index $i$, and $c_i$ is the "component" of index $i$).
For some fixed time $t$,
we make the approximation that $\bvec{w} = W_j \bvec{\Phi}_j$ (Einstein summation)
where $\bvec{W} \in \mathbb{R}^N$ ($N$ is the number of DoFs).
Furthermore, we write that $\frac{\partial \bvec{w}}{\partial t} = X_j \bvec{\Phi}_j$
for some $\bvec{X} \in \mathbb{R}^N$. Then, we may write:
\begin{align*}
    v & = \bvec{w} \cdot \hat{\brvec{e}}_1 \\
        & = W_j \varphi_{b_j} \delta_{1,c_j} \\
    u_\text{e} & = \bvec{w} \cdot \hat{\brvec{e}}_2 \\
        & = W_j \varphi_{b_j} \delta_{2,c_j} \\
    \frac{\partial v}{\partial t} & = \frac{\partial \bvec{w}}{\partial t} \cdot \hat{\brvec{e}}_1 \\
        & = X_j \varphi_{b_j} \delta_{1,c_j}
\end{align*}

Equation \ref{eq1} becomes:
\begin{align*}
    \chi C_\text{m} X_j \varphi_{b_j} \delta_{1,c_j}
            + \chi I_\text{ion}(W_j \varphi_{b_j} \delta_{1,c_j})
        = \nabla \cdot (\sigma_\text{i} \nabla (W_j \varphi_{b_j} \delta_{1,c_j} + W_j \varphi_{b_j} \delta_{2,c_j}))
            + f
\end{align*}
\noindent And, because $\delta_{1,c_j} + \delta_{2,c_j} = 1$:
\begin{align*}
    \chi C_\text{m} X_j \varphi_{b_j} \delta_{1,c_j}
            + \chi I_\text{ion}(W_j \varphi_{b_j} \delta_{1,c_j})
        = \nabla \cdot (\sigma_\text{i} \nabla (W_j \varphi_{b_j}))
            + f
\end{align*}
Similarly, Equation \ref{eq2} becomes:
\begin{align*}
    0 & = \nabla \cdot (\sigma_\text{i} \nabla (W_j \varphi_{b_j} \delta_{1,c_j} + W_j \varphi_{b_j} \delta_{2,c_j}))
            + \nabla \cdot (\sigma_\text{e} \nabla (W_j \varphi_{b_j} \delta_{2,c_j}))
            + f_\text{e} \\
        & = \nabla \cdot (\sigma_\text{i} \nabla (W_j \varphi_{b_j}))
            + \nabla \cdot (\sigma_\text{e} \nabla (W_j \varphi_{b_j} \delta_{2,c_j}))
            + f_\text{e}
\end{align*}

Next, we multiply by a test function $\bvec{\Phi}_i$ and integrate:
\begin{align*}
    & \left< \bvec{\Phi}_i \mid \begin{matrix}
            \chi C_\text{m} X_j \varphi_{b_j} \delta_{1,c_j} + \chi I_\text{ion}(W_j \varphi_{b_j} \delta_{1,c_j}) \\
            0
        \end{matrix} \right> \\
    & \qquad = \left< \bvec{\Phi}_i \mid \begin{matrix}
            \nabla \cdot (\sigma_\text{i} \nabla (W_j \varphi_{b_j})) \\
            \nabla \cdot (\sigma_\text{i} \nabla (W_j \varphi_{b_j})
                + \sigma_\text{e} \nabla (W_j \varphi_{b_j} \delta_{2,c_j}))
        \end{matrix} \right>
        + \left< \bvec{\Phi}_i \mid \begin{matrix}
            f \\
            f_\text{e}
        \end{matrix} \right>
\end{align*}
\noindent Since $\bvec{\Phi}_i(\bvec{x}) = \varphi_{b_i} \hat{\brvec{e}}_{c_i}$:
\begin{align*}
    & \chi C_\text{m} X_j \left< \varphi_{b_i} \mid \varphi_{b_j} \right> \delta_{1,c_i} \delta_{1,c_j}
            + \frac{\chi}{R_\text{m}} W_j \left< \varphi_{b_i} \mid \varphi_{b_j} \right> \delta_{1,c_i} \delta_{1,c_j} \\
        & \qquad = \sigma_\text{i} W_j \left< \varphi_{b_i} \mid \nabla \cdot \nabla \varphi_{b_j} \right> \delta_{1,c_i} \\
        & \qquad + \sigma_\text{i} W_j \left< \varphi_{b_i} \mid \nabla \cdot \nabla \varphi_{b_j} \right> \delta_{2,c_i}
            + \sigma_\text{e} W_j \left< \varphi_{b_i} \mid \nabla \cdot \nabla \varphi_{b_j} \right> \delta_{2,c_i} \delta_{2,c_j} \\
        & \qquad + \left< \varphi_{b_i} \mid f \right> \delta_{1,c_i}
            + \left< \varphi_{b_i} \mid f_\text{e} \right> \delta_{2,c_i}
\end{align*}
\noindent Then, since $\left< \varphi_{b_i} \mid \nabla \cdot \nabla \varphi_{b_j} \right>
    = - \left< \nabla \varphi_{b_i} \mid \nabla \varphi_{b_j} \right>$,
and since $\delta_{1,c_i} + \delta_{2,c_i} = 1$:
\begin{align*}
    & \chi C_\text{m} X_j \left< \varphi_{b_i} \mid \varphi_{b_j} \right> \delta_{1,c_i} \delta_{1,c_j}
            = - \frac{\chi}{R_\text{m}} W_j \left< \varphi_{b_i} \mid \varphi_{b_j} \right> \delta_{1,c_i} \delta_{1,c_j} \\
        & \qquad - \sigma_\text{i} W_j \left< \nabla \varphi_{b_i} \mid \nabla \varphi_{b_j} \right>
            - \sigma_\text{e} W_j \left< \nabla \varphi_{b_i} \mid \nabla \varphi_{b_j} \right> \delta_{2,c_i} \delta_{2,c_j} \\
        & \qquad + \left< \varphi_{b_i} \mid f \right> \delta_{1,c_i}
            + \left< \varphi_{b_i} \mid f_\text{e} \right> \delta_{2,c_i}
\end{align*}
\noindent So, denote:
\begin{align*}
    A_{ij} & = -\frac{\chi}{R_\text{m}} \left< \varphi_{b_i} \mid \varphi_{b_j} \right> \delta_{1,c_i} \delta_{1,c_j}
        - (\sigma_\text{i} + \sigma_\text{e} \delta_{2,c_i} \delta_{2,c_j}) \left< \nabla \varphi_{b_i} \mid \nabla \varphi_{b_j} \right> \\
    M_{ij} & = \chi C_\text{m} \left< \varphi_{b_i} \mid \varphi_{b_j} \right> \delta_{1,c_i} \delta_{1,c_j} \\
    f_i & = \left< \varphi_{b_i} \mid f \right> \delta_{1,c_i} + \left< \varphi_{b_i} \mid f_\text{e} \right> \delta_{2,c_i}
\end{align*}
\noindent Then:
\begin{align} \label{mateq}
    \bmat{M} \bvec{X} = \bmat{A} \bvec{W} + \bvec{f}
\end{align}
\noindent We note that:
1) Although we have formulated the problem as an affine equation,
this is only the case because $I_\text{ion}(v) = \frac{1}{R_\text{m}} v$,
and if we were not using a passive cell model,
we may not be able to represent $\bmat{A}$ as a matrix;
2) $\bmat{M}$ and (in this case) $\bmat{A}$ are constant matrices,
while $\bvec{X}$, $\bvec{W}$, and $\bvec{f}$ each depend on the current time step.

\newpage

\section{Time Integration}

Let us discretize the time domain $(0, T)$ into $k$ fixed-size time steps,
so denote the step size $h = \frac{T}{k}$.
We make the approximation:
\begin{align*}
    \frac{\partial \bvec{w}_{t + 1}}{\partial t} = \frac{\bvec{w}_{t + 1} - \bvec{w}_t}{h}
\end{align*}
\noindent Then, recall that we have already made the approximation:
\begin{align*}
    \bvec{w} & = W_j \varphi_j \\
    \frac{\partial \bvec{w}}{\partial t} & = X_j \varphi_j
\end{align*}
\noindent This implies:
\begin{align*}
    X_{t + 1,j} \varphi_j = \frac{W_{t + 1,j} - W_{t,j}}{h} \varphi_j
\end{align*}
\noindent To keep these approximations consistent, we may simply define:
\begin{align*}
    \bvec{X}_{t + 1} = \frac{1}{h}(\bvec{W}_{t + 1} - \bvec{W}_t)
\end{align*}

We wish to integrate using the Crank-Nicolson scheme,
which we apply to Equation \ref{mateq} as follows:
\begin{align*}
    \frac{1}{h} \bmat{M} (\bvec{W}_{t + 1} - \bvec{W}_t)
        & = \frac{1}{2} (\bmat{A} \bvec{W}_t + \bvec{f}_t)
            + \frac{1}{2} (\bmat{A} \bvec{W}_{t + 1} + \bvec{f}_{t + 1}) \\
    \bmat{M} (\bvec{W}_{t + 1} - \bvec{W}_t)
        & = \frac{h}{2} \bmat{A} (\bvec{W}_{t + 1} + \bvec{W}_t)
            + \frac{h}{2} (\bvec{f}_{t + 1} + \bvec{f}_t)
\end{align*}

As mentioned in the previous section, although this equation is affine,
there are cases in which the vector function $\bmat{A}$ may be nonlinear;
so, we will use Newton's method to solve for $\bvec{W}_{t + 1}$. So, define:
\begin{align*}
    \bvec{R}_{t + 1}(\bvec{W})
        = \bmat{M}(\bvec{W} - \bvec{W}_t)
            - \frac{h}{2} \bmat{A} (\bvec{W} + \bvec{W}_t)
            - \frac{h}{2} (\bvec{f}_{t + 1} + \bvec{f}_t)
\end{align*}
\noindent Where $\bvec{W}_t$ above has been determined by the initial conditions
or by the previous time step. The iteration proceeds as follows:
\begin{align*}
    \bvec{W}_{t + 1}^{(0)} & = \bvec{W}_t \\
    \brmat{J}[\bvec{R}_{t + 1}] \Delta \bvec{W}_{t + 1}^{(\ell + 1)} & = - \bvec{W}_{t + 1}^{(\ell)} \\
    \bvec{W}_{t + 1}^{(\ell + 1)} & = \bvec{W}_{t + 1}^{(\ell)} + \alpha_{t + 1} \Delta \bvec{W}_{t + 1}^{(\ell + 1)}
\end{align*}
\noindent Where $\brmat{J}[\bvec{R}_{t + 1}]$ is the Jacobian of $\bvec{R}_{t + 1}$
and $\alpha_{t + 1}$ is a step size to be chosen by the Newton solver.
Once the residual $||\bvec{W}_{t + 1}^{(\ell)}||$ is sufficiently small,
we take $\bvec{W}_{t + 1} = \bvec{W}_{t + 1}^{(\ell)}$ as the solution for the time step.

\newpage

\section{Linear Formulation}

We can also solve the problem using the assumption that $I_\text{ion}(v) = \frac{1}{R_\text{m}} v$.
In this case, the Bidomain equations become linear, allowing us to forego Newton iteration.
If we solve using Backward Euler, we have that:
\begin{align*}
    \frac{1}{h} \bmat{M} (\bvec{W}_{t + 1} - \bvec{W}_t) & = \bmat{A} \bvec{W}_{t + 1} + \bvec{f}_{t + 1} \\
    (\frac{1}{h} \bmat{M} - \bmat{A}) \bvec{W}_{t + 1} & = \frac{1}{h} \bmat{M} \bvec{W}_t + \bvec{f}_{t + 1}
\end{align*}
\noindent Or, using Crank-Nicolson:
\begin{align*}
    \frac{1}{h} \bmat{M} (\bvec{W}_{t + 1} - \bvec{W}_t)
        & = \frac{1}{2} (\bmat{A} (\bvec{W}_{t + 1} + \bvec{W}_t) + \bvec{f}_{t + 1} + \bvec{f}_t) \\
    (\frac{1}{h} \bmat{M} - \frac{1}{2} \bmat{A}) \bvec{W}_{t + 1}
        & = (\frac{1}{h} \bmat{M} + \frac{1}{2} \bmat{A}) \bvec{W}_t + \frac{1}{2} (\bvec{f}_{t + 1} + \bvec{f}_t)
\end{align*}

\newpage

\section{Godunov Splitting}

Let us slightly rewrite the original system of equations:
\begin{align*}
    \chi C_\text{m} \frac{\partial v}{\partial t}
        & = \nabla \cdot (\sigma_i \nabla (v + u_\text{e})) - \chi I_\text{ion}(v) + f \\
    0
        & = \nabla \cdot (\sigma_i \nabla (v + u_\text{e})) + \nabla \cdot (\sigma \nabla u_\text{e}) + f_\text{e}
\end{align*}
\noindent Which, after spatial discretization, has been written as:
\begin{align*}
    \bmat{M} \bvec{X} = \bmat{A} \bvec{W} + \bvec{f}
\end{align*}
\noindent Where $\bmat{M}$ is the mass matrix,
$\bmat{A}$ is the stiffness matrix,
$\bvec{f}$ is the stiffness right-hand side vector,
$\bvec{W}$ is the solution state vector,
and $\bvec{X}$ is the state vector for the solution's time derivative
(which is typically written as $\bvec{X}_{t + 1} = \frac{1}{h}(\bvec{W}_{t + 1} - \bvec{W}_t)$).

We split the right-hand side operator as follows:
\begin{align*}
    \bmat{M} \bvec{X} & = \bvec{f} \\
    \bmat{M} \bvec{X} & = \bmat{A} \bvec{W}
\end{align*}
And, applying Backward Euler time stepping to each system, we get:
\begin{align*}
    \bmat{M} \bvec{X}_t^{(*)} & = \bvec{f}_{t + 1} \\
    \bmat{M} \bvec{X}_{t + 1} & = \bvec{A} \bvec{W}_{t + 1}
\end{align*}
\noindent Substituting $\bvec{X}_t^{(*)} = \frac{1}{h} (\bvec{W}_t^{(*)} - \bvec{W}_t)$,
we get:
\begin{align*}
    \frac{1}{h} \bmat{M} (\bvec{W}_t^{(*)} - \bvec{W}_t) & = \bvec{f}_{t + 1} \\
    \frac{1}{h} \bmat{M} \bvec{W}_t^{(*)} & = \frac{1}{h} \bmat{M} \bvec{W}_t + \bvec{f}_{t + 1}
\end{align*}
\noindent And substituting $\bvec{X}_{t + 1} = \frac{1}{h} (\bvec{W}_{t + 1} - \bvec{W}_t^{(*)})$,
we get:
\begin{align*}
    \frac{1}{h} \bmat{M} (\bvec{W}_{t + 1} - \bvec{W}_t^{(*)}) & = \bmat{A} \bvec{W}_{t + 1} \\
    (\frac{1}{h} \bmat{M} - \bmat{A}) \bvec{W}_{t + 1} & = \frac{1}{h} \bmat{M} \bvec{W}_t^{(*)}
\end{align*}
\noindent So, our time stepping scheme is:
\begin{align*}
    \frac{1}{h} \bmat{M} \bvec{W}_t^{(*)} & = \frac{1}{h} \bmat{M} \bvec{W}_t + \bvec{f}_{t + 1} \\
    (\frac{1}{h} \bmat{M} - \bmat{A}) \bvec{W}_{t + 1} & = \frac{1}{h} \bmat{M} \bvec{W}_t^{(*)}
\end{align*}

\newpage

\section{Version History}

The \texttt{bidomain[\_2]} targets are an attempt to solve the Bidomain problem
assuming that the equations may be nonlinear. However, the version using
the data described in "Toy Problem" performs poorly;
in particular, the $u_\text{e}$ component is forced to $0$,
and while the shape of the $v$ component is correct,
it appears scaled down.
However, when solving the problem according to the method described
in "Linear Formulation", these problems do not arise.
\texttt{bidomain\_godunov[\_2]} does Godunov splitting using this
linear method, which performs similarly well.

The \texttt{*\_2} targets use a different toy problem in which
$v$ and $u_\text{e}$ are linearly independent,
and the \texttt{bidomain\_2} performs better on this problem
(although the scaling of both $v$ and $u_\text{e}$ are slightly off,
the $u_\text{e}$ component is not forced to $0$).
The \texttt{bidomain\_godunov\_2} code performs
about as well as \texttt{bidomain\_godunov}.

\end{document}
