\documentclass{article}

\usepackage{amsfonts}
\usepackage{amsmath}

\newcommand{\bvec}[1]{\boldsymbol{#1}}
\newcommand{\brvec}[1]{\mathbf{#1}}
\newcommand{\bmat}[1]{\boldsymbol{#1}}
\newcommand{\brmat}[1]{\mathbf{#1}}
\newcommand{\ii}{\mathrm{i}}
\newcommand{\ee}{\mathrm{e}}
\newcommand{\dd}{\mathrm{d}}

\begin{document}

\section{Introduction}

Let $\Omega \subseteq \mathbb{R}^n$,
$V : \Omega \rightarrow \mathbb{R}$,
$\kappa \in \mathbb{R}$,
and $\psi : (0, T) \times \Omega \rightarrow \mathbb{C}$.
The Nonlinear Schroedinger Equation (NSE) is given by:
\begin{align*}
-\ii \frac{\partial \psi}{\partial t} - \frac{1}{2} \Delta \psi + V \psi + \kappa |\psi|^2 \psi = 0
\end{align*}
\noindent With initial conditions:
\begin{align*}
\psi(0, \bvec{x}) = \psi_0(\bvec{x})
\end{align*}
\noindent For all $\bvec{x} \in \Omega$ and with boundary conditions:
\begin{align*}
\psi(t, \bvec{x}) = 0
\end{align*}
\noindent For all (t, $\bvec{x}) \in (0, T) \times \partial \Omega$.

Because $\psi$ is complex-valued, it is convenient to separate the equation into real and imaginary parts.
If we take $\psi(t, \bvec{x}) = u(t, \bvec{x}) + \ii v(t, \bvec{x})$
for $u, v : (0, T) \times \Omega \rightarrow \mathbb{R}$,
we also have that $|\psi|^2 = u^2 + v^2$:
\begin{align*}
    -\ii \frac{\partial}{\partial t} (u + \ii v) - \frac{1}{2} \Delta (u + \ii v)
            + V (u + \ii v) + \kappa (u^2 + v^2) (u + \ii v)
        = 0
\end{align*}
\noindent So:
\begin{align*}
    \frac{\partial v}{\partial t} - \frac{1}{2} \Delta u + V u + \kappa (u^2 + v^2) u & = 0 \\
    -\frac{\partial u}{\partial t} - \frac{1}{2} \Delta v + V v + \kappa (u^2 + v^2) v & = 0
\end{align*}

\newpage

\section{Toy Problem}

Let $\Omega = [-1, 1]^n$, and consider, for an example:
\begin{align*}
    u(t, \bvec{x}) & = \cos(\pi t) \prod_{i = 1}^n \cos(\pi x_i) \\
    v(t, \bvec{x}) & = \sin(\pi t) \prod_{i = 1}^n \cos(\pi x_i)
\end{align*}
\noindent Then:
\begin{align*}
    u(0, \bvec{x}) & = \prod_{i = 1}^n \cos(\pi x_i) \\
    v(0, \bvec{x}) & = 0
\end{align*}

We have that:
\begin{align*}
    u^2 + v^2
        & = \cos^2(\pi t) \prod_{i = 1}^n \cos^2(\pi x_i)
            + \sin^2(\pi t) \prod_{i = 1}^n \cos^2(\pi x_i) \\
        & = (\cos^2(\pi t) + \sin^2(\pi t)) \prod_{i = 1}^n \cos^2(\pi x_i) \\
        & = \prod_{i = 1}^n \cos^2(\pi x_i)
\end{align*}
\noindent Note that $u^2 + v^2$ is independent of $t$. So:

Meanwhile:
\begin{align*}
    \frac{\partial u}{\partial x_k} & = -\pi \cos(\pi t) \sin(\pi x_k) \prod_{i \neq k} \cos(\pi x_i) \\
    \frac{\partial^2 u}{\partial x_k^2} & = -\pi^2 \cos(\pi t) \prod_{i = 1}^n \cos(\pi x_i) \\
        & = -\pi^2 u \\
    \Delta u & = \sum_{k = 1}^n \frac{\partial^2 u}{\partial x_k^2} \\
        & = -n \pi^2 u
\end{align*}
\noindent Similarly:
\begin{align*}
    \Delta v = -n \pi^2 v
\end{align*}
\noindent And:
\begin{align*}
    \frac{\partial u}{\partial t} & = -\pi \sin(\pi t) \prod_{i = 1}^n \cos(\pi x_i) \\
        & = -\pi v \\
    \frac{\partial v}{\partial t} & = \pi \cos(\pi t) \prod_{i = 1}^n \cos(\pi x_i) \\
        & = \pi u
\end{align*}

Then, rewriting the first equation:
\begin{align*}
    V u & = -\frac{\partial v}{\partial t} + \frac{1}{2} \Delta u - \kappa (u^2 + v^2) u \\
        & = -\pi u - \frac{n}{2} \pi^2 u - \kappa (u^2 + v^2) u \\
    V(\bvec{x}) & = -\pi - \frac{n}{2} \pi^2 - \kappa (u^2 + v^2)
\end{align*}
\noindent And the second equation:
\begin{align*}
    V v & = \frac{\partial u}{\partial t} + \frac{1}{2} \Delta v - \kappa (u^2 + v^2) v \\
        & = -\pi v - \frac{n}{2} \pi^2 v - \kappa (u^2 + v^2) v \\
    V(\bvec{x}) & = -\pi - \frac{n}{2} \pi^2 - \kappa (u^2 + v^2)
\end{align*}
\noindent We see that the equations agree.

\newpage

\section{Spatial Discretization}

Consider a set of basis function $\bvec{\Phi}_i : \Omega \rightarrow \mathbb{R}^2$,
where each $\bvec{\Phi}_i(\bvec{x}) = \varphi_{b_i} \hat{\brvec{e}}_{c_i}$,
where $\varphi_j : \Omega \rightarrow \mathbb{R}$ is a set of basis functions
($b_i$ is the base index of the index $i$ and $c_i$ is the component index of $i$).
We write $\bvec{w} = \begin{bmatrix}
    u \\
    v
\end{bmatrix}$ and make the approximation $\bvec{w} = W_j \bvec{\Phi}_j$ (Einstein summation).
Then, we multiply and integrate the system with a basis function $\bvec{\Phi}_i$:
\begin{align*}
    \left< \bvec{\Phi}_i \mid \begin{matrix}
        \ii \frac{\partial v}{\partial t} - \frac{1}{2} \Delta u + V u + \kappa (u^2 + v^2) u \\
        -\ii \frac{\partial u}{\partial t} - \frac{1}{2} \Delta v + V v + \kappa (u^2 + v^2) v
    \end{matrix} \right> = 0
\end{align*}

\end{document}
