\documentclass{article}

\usepackage{amsfonts}
\usepackage{amsmath}

\newcommand{\bvec}[1]{\boldsymbol{#1}}
\newcommand{\brvec}[1]{\mathbf{#1}}
\newcommand{\bmat}[1]{\boldsymbol{#1}}
\newcommand{\brmat}[1]{\mathbf{#1}}
\newcommand{\ii}{\mathrm{i}}
\newcommand{\ee}{\mathrm{e}}
\newcommand{\dd}{\mathrm{d}}

\begin{document}

\section{Introduction}

Consider a spatial domain $\Omega \subseteq \mathbb{R}^n$
and a time domain $(0, T)$ for some $T \in \mathbb{R}$.
The well-known heat equation is given as:
\begin{align*}
    \frac{\partial u}{\partial t} = \Delta u + f
\end{align*}
\noindent Where $u, f : (0, T) \times \Omega \rightarrow \mathbb{R}$,
with initial conditions:
\begin{align*}
    u(0, \bvec{x}) = u_0(\bvec{x})
\end{align*}
\noindent For all $\bvec{x} \in \Omega$.
We may pose Dirichlet boundary conditions of the form:
\begin{align*}
    u(t, \bvec{x}) = g(t, \bvec{x})
\end{align*}
\noindent Or we may pose Neumann boundary conditions of the form:
\begin{align*}
    \hat{\brvec{n}} \cdot \nabla u = h(t, \bvec{x})
\end{align*}
\noindent For all $(t, \bvec{x}) \in (0, T) \times \partial \Omega$.

We will be considering the case of a square L-shaped domain in $\mathbb{R}^n$,
i.e., a domain of the form $\Omega = (-1, 1)^n \setminus (0, 1)^n$.
This is a hyper-cube domain missing the hyper-cube around the "rightmost"
(i.e., most positive in each repective dimension) corner.
Our heat source $f$ will be located at the center of the domain;
furthermore, we will split the domain into interior and exterior sections
and split the heat operator, treating the two sections separately.

\newpage
\section{Domain Splitting}

We split the domain $\Omega$ into interior and exterior sections; repectively:
\begin{align*}
    \Omega_\text{I} & = (-1/2, 1/2)^n \setminus (0, 1/2)^n \\
    \Omega_\text{E} & = \Omega \setminus \overline{\Omega}_\text{I}
\end{align*}
\noindent We see that $\Omega_\text{I}$ is itself an L-shaped domain,
and $\Omega_\text{E}$ is the shell around it.
When we consider the heat operator separately on these two sections,
we must consider the boundary $\Gamma_\text{I} = \partial \Omega_\text{I}$ which separates them.
Clearly, we must have, for all fixed $t \in (0, T)$ and $\bvec{x} \in \partial \Omega_\text{I}$:
\begin{align*}
    \lim_{\Omega_\text{I} \ni \bvec{\xi} \rightarrow \bvec{x}} u(t, \bvec{\xi})
        & = \lim_{\Omega_\text{E} \ni \bvec{\xi} \rightarrow \bvec{x}} u(t, \bvec{\xi}) \\
    \lim_{\Omega_\text{I} \ni \bvec{\xi} \rightarrow \bvec{x}} \hat{\brvec{n}} \cdot \nabla u(t, \bvec{\xi})
        & = -\lim_{\Omega_\text{E} \ni \bvec{\xi} \rightarrow \bvec{x}} \hat{\brvec{n}} \cdot \nabla u(t, \bvec{\xi})
\end{align*}

\newpage
\section{Spatial Discretization}

For a fixed time $t \in (0, T)$,
we make the approximation $u = U_j \varphi_j$ (Einstein summation)
for a set of basis functions $\varphi_j : \Omega \rightarrow \mathbb{R}$;
similarly, we write $\frac{\partial u}{\partial t} = X_j \varphi_j$.
So, the original equation becomes:
\begin{align*}
    X_j \varphi_j = U_j \Delta \varphi_j + f
\end{align*}
\noindent We multiply by a test function $\varphi_i$ and integrate:
\begin{align*}
    X_j \left< \varphi_i \mid \varphi_j \right>
        = U_j \left< \varphi_i \mid \Delta \varphi_j \right>
            + \left< \varphi_i \mid f \right>
\end{align*}

Consider:
\begin{align*}
    \left< \varphi_i \mid \Delta \varphi_j \right>
        & = \int_\Omega \varphi_i \Delta \varphi_j \, \dd \bvec{x} \\
        & = \oint_{\partial \Omega} \varphi_i \hat{\brvec{n}} \cdot \nabla \varphi_j \, \dd \bvec{s}
            - \int_\Omega \nabla \varphi_i \cdot \nabla \varphi_j \, \dd \bvec{x} \\
        & = \left< \varphi_i \mid \hat{\brvec{n}} \cdot \nabla \varphi_j \right>_{\partial \Omega}
            - \left< \nabla \varphi_i \mid \nabla \varphi_j \right>
\end{align*}

\end{document}