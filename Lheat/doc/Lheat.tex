\documentclass{article}

\usepackage{amsfonts}
\usepackage{amsmath}

\newcommand{\bvec}[1]{\boldsymbol{#1}}
\newcommand{\brvec}[1]{\mathbf{#1}}
\newcommand{\bmat}[1]{\boldsymbol{#1}}
\newcommand{\brmat}[1]{\mathbf{#1}}
\newcommand{\ii}{\mathrm{i}}
\newcommand{\ee}{\mathrm{e}}
\newcommand{\dd}{\mathrm{d}}

\begin{document}

\section{Introduction}

Consider a spatial domain $\Omega \subseteq \mathbb{R}^n$
and a time domain $(0, T)$ for some $T \in \mathbb{R}$.
The well-known heat equation is given as:
\begin{align*}
    \frac{\partial u}{\partial t} = \Delta u + f
\end{align*}
\noindent Where $u, f : (0, T) \times \Omega \rightarrow \mathbb{R}$,
with initial conditions:
\begin{align*}
    u(0, \bvec{x}) = u_0(\bvec{x})
\end{align*}
\noindent For all $\bvec{x} \in \Omega$.
We may pose Dirichlet boundary conditions of the form:
\begin{align*}
    u(t, \bvec{x}) = g(t, \bvec{x})
\end{align*}
\noindent Or we may pose Neumann boundary conditions of the form:
\begin{align*}
    \hat{\brvec{n}} \cdot \nabla u = h(t, \bvec{x})
\end{align*}
\noindent For all $(t, \bvec{x}) \in (0, T) \times \partial \Omega$.

We will be considering the case of a square L-shaped domain in $\mathbb{R}^n$,
i.e., a domain of the form $[-1, 1]^n \setminus [0, 1]^n$.
This is a hyper-cube domain missing the hyper-cube around the "rightmost"
(i.e., most positive in each repective dimension) corner.
Furthermore, we will split the domain into interior and exterior sections
and split the heat equation operator, treating the two sections separately.

\newpage
\section{Domain Splitting}



\end{document}